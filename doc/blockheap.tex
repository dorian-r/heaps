\documentclass[10pt, a4paper]{article}

\usepackage[english]{babel}
\usepackage[margin=2cm, nohead]{geometry}
\usepackage{parskip}
\usepackage{listings}
\usepackage{array}
\usepackage{arydshln}
\usepackage[T1]{fontenc}
\usepackage{amsmath}
\usepackage{graphicx}
\usepackage{tikz}
\usetikzlibrary{shapes,decorations.pathreplacing,positioning,bending,calc}
%\usetikzlibrary{shapes.multipart}


\begin{document}

\begin{center}
\LARGE Block Heap Layout
\end{center}

\section{Abstract}

In this document I will present an alternative memory layout for binary heaps called block heap layout (BHL).
In a $d$-BHL subtrees of height $d$ are stored continuously opposed to the standard layout for binary trees which stores nodes layer by layer.
I will first define the BHL structure and then analyse it in comparison to the standard binary tree layout.

\section{Definition}

The following illustration represents one block in the $d$-BHL structure (the number in the node refers to its position when saved in the array).

\begin{tikzpicture}[level/.style={sibling distance=3cm/#1,level distance=1cm}]
\node [rectangle,draw] (n0){0}
	child {node [rectangle,draw] (n1) {1}
		child {node{$\vdots$}
			child {node [rectangle,draw,left] (n3) {$2^d-1$}}
		}
		child {node{$\vdots$}}
	}
	child {node [rectangle,draw] (n2) {2}
		child {node{$\vdots$}}
		child {node{$\vdots$}
			child {node [rectangle,draw,right] (n4) {$2^{d+1}-2$}}
		}
	};
    
\path (n3) -- (n4) node [midway] {$\cdots$};

\coordinate (cd1) at ($(n4)+(1.5,0)$);
\draw[<->,] 
(cd1) -- (cd1|-n0.east) node [midway,fill=white] {$d$};
\end{tikzpicture}

Let $D=2^{d+1}$. In order to use the BHL structure for heaps of any size, we can make each of the $D$ children of leafs in a block root nodes of another block. Thus a $D$-ary tree is formed by the blocks.

\begin{tikzpicture}[
	level/.style={sibling distance=3cm/#1,level distance=1cm},
	child anchor=north,parent anchor=south,
	every text node part/.style={align=center},
	nd/.style={isosceles triangle,draw,shape border rotate=90, isosceles triangle apex angle=60,inner sep=0,font=\tiny,yshift={-20mm},text width=2cm},
]
\node[nd] {$0\ldots D-2$}
	child{node[nd,left] {$D\ldots 2D-2$}}
	child{node[nd,left] (a){$2D\ldots 3D-2$}}
	child{node[nd,right] (b){$D^2\ldots D^2+D-2$}};
\path (a) -- (b) node [midway] {$\cdots$};
\end{tikzpicture}

\end{document}