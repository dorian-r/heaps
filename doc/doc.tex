\documentclass[10pt, a4paper]{article}

\usepackage[english]{babel}
\usepackage[margin=2cm, nohead]{geometry}
\usepackage{parskip}
\usepackage{listings}
\usepackage{array}
\usepackage{arydshln}
\usepackage[T1]{fontenc}
\usepackage{amsmath}
\usepackage{graphicx}

\title{Extended Radix Heap}
\date{}

\begin{document}

\maketitle

\section{Correction}

Before addressing the implementation I have to make one small correction to the given \texttt{deleteMin} algorithm from the slides: When $B[-1]$ is not empty, we have to remove the first element and join its buckets with the heap if it is a super element until a normal element is removed. Since super elements will have to be joined to the heap at some point this does not effect the amortized time complexity.

If we would search for a normal element with each \texttt{deleteMin}, an example could easily be constructed with $n/2$ super elements of size 1 followed by $n/2$ normal elements in $B[-1]$. Then removing all $n$ elements would take $\Omega(n^2)$ time. 

\section{Efficient implementation}

\end{document}